\documentclass[sigconf,anonymous]{acmart}
%\documentclass[sigconf]{acmart}
\begin{document}

\acmConference[ITiCSE'18]{Innovation and Technology in Computer Science Education}{July 2--4, 2018}{Larnaca, Cyprus}

\title{Preparing First-Time CS Student Teaching Assistants}

\author{Guido R\"o\ss{}ling}
\email{roessling.gu@kiva.tu-darmstadt.de}
\author{Jacqueline G\"olz}
\email{goelz.ja@kiva.tu-darmstadt.de}
\affiliation[obeypunctuation=true]{
  \institution{TU Darmstadt, CS Dept.}\\
  \streetaddress{Hochschulstr. 10}
  \postcode{64289}
  \city{Darmstadt},
  \country{Germany}
}
\begin{CCSXML}
<ccs2012>
<concept>
<concept_id>10003456.10003457.10003527.10003531.10003533</concept_id>
<concept_desc>Social and professional topics~Computer science education</concept_desc>
<concept_significance>500</concept_significance>
</concept>
</ccs2012>
\end{CCSXML}
\setcopyright{rightsretained}

\ccsdesc[500]{Social and professional topics~Computer science education}
\keywords{student TAs, training, teaching}


\begin{abstract}
Student Teaching Assistants (TAs) are an important help, especially in large
undergraduate courses. They act as role models and as agents between students and teachers.
They also help guide the students' learning process by supervising groups and grading homework.
We prepare new TAs by training their diagnostic and didactic skills.
\end{abstract}

\maketitle

\section{Preparing CS Student TAs}

Student TAs are often only a few semesters ahead of the students they supervise and support.
They usually lead a weekly exercise group, consisting of about 15-35 students. They are expected to
assist students in understanding and applying the topics of the associated lecture by
working on sample problems. They also grade the weekly homework submissions. Although the handing in of
homework is not strictly monitored, most lectures in our first year of study require a minimum number of
homework points to be achieved in order to be eligible for taking part in the final exam.

We expect our TAs to support the students by allowing them to achieve the exercise goals with as little
direct advice as possible. As many students as possible should be able to reach their
own solution using TA assistance and feedback, but without simply receiving the solution or a step-wise explanation. TAs should employ different intensities of support depending on the current student's
state of progress and understanding, ranging from encouragement (``you can do this'')
up to solution-oriented hints (``try Merge Sort here'').
Questioning techniques (``what did you do so far?'') are re\-commended to identify
the source of their students' problems, to ensure the TA's efforts address the actual needs of the
given student. 

To achieve these goals, our TAs receive a training comparable to the one described in
\cite{Estrada:2017:BGD:3059009.3059023} before their first group meeting,
which depending on the concrete lecture takes between four and 16 hours, the latter split over two full days.
Typically, there are also weekly meetings between the TAs and the lecturers or their assistants, where 
the upcoming exercise sheet and current issues are discussed.

During this training, TAs are prepared for their job by...
\begin{itemize}
\item reflecting individually and as a group on their tasks, and what characterizes a
  ``great'' or ``weak'' student TA,
\item reflecting on and developing appropriate assisting strategies,
\item preparing for a smooth start into the first exercise group: what information needs to be provided?
How to best announce my name, contact data, and office hours?
\item simulating challenging teaching situations. For example, one person acts as the TA responding
to a student complaining about the grading of the homework, while other students struggle with the current exercise,

%, while the others
%create a challenging---but realistic---teaching situation, e.g., a heated debate about the
%points awarded for the homework binding the tutor's attention, while other students struggle with the current exercise,
\item understanding the fundamentals of group communication, teamwork, and avoiding ambiguities,
\item becoming familiar with grading homework and providing sensible and respectful feedback,
\item dealing sensibly with potential cases of academic dishonesty,
\item using the department's Moodle platform as a student TA.
\end{itemize}

The training combines different types of activities, ranging from a brief presentation,
think-pair-share activities and role-playing to individual work. 
We also perform work shadowing and reflect 
the observed tutorial together with the student TA in a feedback meeting. 
About 400 student TAs have participated in this training since summer 2014. Both student TA trainings and tutorials
are evaluated. The grades for the trainings are usually between 1 and 2, with 1 being the best grade on a 
1-5 scale.


%\begin{itemize}
%\item Hospitation, experiences - do tutors do a good job? Feedback? More confident/feeling better prepared for their job after the training?...
%\item Nr. of participants...
%\item What else? ...
%\item To Dos...
%\item Bibliographic references, e.g. Student TA Interview ITiCSE'17? ...
%\end{itemize}

\begin{acks}
This work was funded by the German Federal Ministry of Education and Research as part of the 
``Competence Development through Interdisciplinary and International Cooperation from the Start
(K$\mathrm{I^2}$VA)'' project.
\end{acks}

\bibliographystyle{ACM-Reference-Format}
\bibliography{iticse18}
\end{document}
