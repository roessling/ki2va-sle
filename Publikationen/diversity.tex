%\documentclass[sigconf,anonymous]{acmart}
\documentclass[sigconf,screen]{acmart}
\begin{document}

\acmConference[ITiCSE'18]{Innovation and Technology in Computer Science Education}{July 2--4, 2018}{Larnaca, Cyprus}

\title{An Interactive Online Course to Raise Awareness about Diversity}

\author{Guido R\"o\ss{}ling}
\email{roessling@acm.org}
\affiliation[obeypunctuation=true]{
  \institution{TU Darmstadt, CS Dept.}\\
  \streetaddress{Hochschulstr. 10}
  \postcode{64289}
  \city{Darmstadt},
  \country{Germany}
}
\author{Felix Broj}
\email{f.broj@apaed.tu-darmstadt.de}
\affiliation[obeypunctuation=true]{
  \institution{TU Darmstadt, Pedagogy Dept.}\\
  \streetaddress{Alexanderstr. 23}
  \postcode{64289}
  \city{Darmstadt},
  \country{Germany}
}
\begin{CCSXML}
<ccs2012>
<concept>
<concept_id>10003456.10003457.10003527.10003531.10003533</concept_id>
<concept_desc>Social and professional topics~Computer science education</concept_desc>
<concept_significance>500</concept_significance>
</concept>
</ccs2012>
\end{CCSXML}

\ccsdesc[500]{Social and professional topics~Computer science education}
\keywords{online course, interactive, self-training, diversity, awareness}

\begin{abstract}
Diversity is sometimes understood narrowly, by focusing on only one or two categories,
such as gender or race. Following the concept of intersectionality, there are far more
aspects to diversity that can affect students' success, in some cases only indirectly or
subtly. We present a small self-contained online course for raising awareness of
some additional aspects of diversity, without admonishing users or prescribing attitudes.
\end{abstract}

\maketitle

\section{An Interactive Online Course about Diversity}

Diversity has, as the name indicates, many different aspects. Gender and racial
background, while arguably the most common aspects, are only facets of diversity. Other
concepts of diversity include, but are not limited to, disabilities, the sense of
national, ethnical and/or cultural identity, religious backgrounds, social and economic
background, sexual disposition, political views, and finally age.

In many cases, courses or programmes seem to expect an ``average student'' to which
certain attributes are ascribed: not having  a disability, a citizen of the university's nation,
adhering to the predominant local religion, coming from a financially stable background,
heterosexual, and of ``typical age'' for a student. In other words, the ``average student''
is essentially the ``average John or Jane'', comparable to ``all others of the same age''.
This is in itself a fallacy, and the reality is of course far more complex and diverse than
one may anticipate.

Our course presents a set of six ``personas'' representing fictional characters. The
personas are not based on actual students, but are assembled based on a large
university survey. Thus, students just like our personas are likely to exist.
We have made sure that the personas diverge to varying degrees from stereotypical expectations
of a ``typical'' student. Each of these personas provides some background information, including the
given name, age, information about their sex, gender and desire, national, ethnic and
cultural background, religion, parents, field of study, and a few lines of additional
background information. In essence, each of these fictional personas
should feel like someone one could meet on campus.

The user is then randomly assigned one of these personas and asked to rate 20
statements for this persona with agreement, undecided or disagreement. Each answer
is assigned 2, 1 or 0 points, thus up to 40 points can be gathered by a student
that perfectly agrees with each statement. The statements were developed from a
related diversity method and indicate---though in several cases subtle---differences
in students' lives on campus, for example based on financial backgrounds, the relationship
to parents, or social obligations that cut into the time available for studying
(or leisure). Both the avatar for the persona assigned to the user  and six additional
other personas---separate from those from which the user persona is chosen---progress
from the bottom of the screen towards the top, depending on the respective answer
score. In this way, the user can easily see that the results for the personas based
on these statements is highly uneven: at the end, the six other personas will have a
score between 13 and 37, with the user's persona typically falling
somewhere between these extremes. The ``answers'' for the six other personas was determined by
averaging the estimations of the likely answer from colleagues and pedagogy students. 

The course closes with a chance to reflect on the different starting points of a diverse
student body. Teaching assistants and academics are requested to make sure that the diversity of our
students--usually through no fault or decision of their own--does not have a negative
impact on their academic success. The course shall also be used in our forthcoming
trainings for student teaching assistants. We hope that the course will encourage students to reflect
on the different characteristics of other students, and open their minds to diversity aspects and the
advantages--and disadvantages--that certain situations or societal expectations convey.

The course is available at \url{https://hidden.for.review} for viewing and commenting. Currently, it is
only available in German, but we plan to provide an English version in time for ITiCSE'18.

%\setcitestyle{acmnumeric}

\begin{acks}
This work was funded by the German Federal Ministry of Education and Research as part of the 
``Competence Development through Interdisciplinary and International Cooperation from the Start
(K$\mathrm{I^2}$VA)'' project.
\end{acks}

\bibliographystyle{ACM-Reference-Format}
%\bibliography{iticse18}
\end{document}
