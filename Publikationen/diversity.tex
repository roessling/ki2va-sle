%\documentclass[sigconf,anonymous]{acmart}
\documentclass[sigconf]{acmart}
\begin{document}

\acmConference[ITiCSE'18]{Innovation and Technology in Computer Science Education}{July 2--4, 2018}{Larnaca, Cyprus}

\title{An Interactive Online Course to Raise Awareness about Diversity}

\author{Guido R\"o\ss{}ling}
\email{roessling@acm.org}
\affiliation[obeypunctuation=true]{
  \institution{TU Darmstadt, CS Dept.}\\
  \streetaddress{Hochschulstr. 10}
  \postcode{64289}
  \city{Darmstadt},
  \country{Germany}
}
\author{Felix Broj}
\email{broj.fe@kiva.tu-darmstadt.de}
\affiliation[obeypunctuation=true]{
  \institution{TU Darmstadt, Pedagogy Dept.}\\
  \streetaddress{Alexanderstr. 23}
  \postcode{64289}
  \city{Darmstadt},
  \country{Germany}
}
\begin{CCSXML}
<ccs2012>
<concept>
<concept_id>10003456.10003457.10003527.10003531.10003533</concept_id>
<concept_desc>Social and professional topics~Computer science education</concept_desc>
<concept_significance>500</concept_significance>
</concept>
</ccs2012>
\end{CCSXML}

\ccsdesc[500]{Social and professional topics~Computer science education}
\keywords{online course, interactive, self-training, diversity, awareness}

\begin{abstract}
Diversity is sometimes understood narrowly, by viewing only gender and racial aspects.
However, there are far more aspects to diversity that can affect the chances to study successfully,
in some cases only indirectly or subtly. We present a small self-contained online course for raising
awareness of some additional aspects of diversity, without admonishing users or prescribing attitudes.
\end{abstract}

\maketitle

\section{An Interactive Online Course about Diversity}

Diversity has, as the name indicates, many different aspects. Gender and racial background, while arguably the most
common aspects, are only facets of diversity. Other aspects include, but are not limited to, disabilities, the sense of
national, ethnical and/or cultural identity, religious backgrounds--both concerning the actual personal belief system
and the degree to which one identifies with central tenets--, social and economical background, sexual disposition, and
finally age.

In many cases, courses or programmes seem to expect an ``average student'' to which certain attributes are ascribed: not having 
a disability, a citizen of the university's nation, adhering to the predominant local religion, coming from a financially
stable background, heterosexual, and of ``typical age'' for a student. In other words, the ``average student'' is essentially your
average John or Jane, comparable to ``all others of the same age''. This is in itself a fallacy, and the reality is of course far
more diverse than anticipated.

Our course presents a set of six personas--not actually persons, but fictional characters that however could easily be found on
any campus. Each of these personas provides information on his or her background: given name, age, information about their
relationship or living status, national, ethnical and cultural background, religion, parents, field of study, and a few lines of additional
background information. Each fictional personal should feel like someone one could meet on campus.

The user is then randomly assigned one of these personas and asked to rate 20 questions for this persona with
agreement, undecided or disagreement. Each answer is awarded 2, 1 or 0 points, thus up to 40 points can be achieved by a student
that perfectly agrees with each statement. The statements are chosen based on a diversity study and indicate--though in several
cases subtly--differences between students, for example based on financial backgrounds, the relationship to parents, or social
obligations that cut into the time available for studying (or leisure). Both the user's persona's portray and six additional
other persona--separate from those from which the user persona is chosen--progress from the bottom of the screen towards the top,
depending on the respective answer score. In this way, the user can easily see that the ``staging area'' for students based on these
questions is highly uneven: at the end, the six other personas will have a (predetermined) score between 13 and 37, with the user's
persona typically falling somewhere between these extremes.

The course closes with a chance to reflect on the different chances for the diverse student body and a request to make sure that
students--usually through no fault or decision of their own--have equal chances. We hope that this will encourage students to reflect 
on the different characteristics of other students, and open their minds to diversity aspects and the advantages--and disadvantages--that
certain expectations or societal expectations convey.

The course is available at \url{URL disclosed for anonymous review} for viewing and commenting to anybody interested.

%\setcitestyle{acmnumeric}

\begin{acks}
This work was funded by the German Federal Ministry of Education and Research as part of the 
``Competence Development through Interdisciplinary and International Cooperation from the Start
(K$\mathrm{I^2}$VA)'' project.
\end{acks}

\bibliographystyle{ACM-Reference-Format}
%\bibliography{iticse18}
\end{document}
