%\documentclass[sigconf,anonymous]{acmart}
\documentclass[sigconf]{acmart}
\begin{document}

\acmConference[ITiCSE'18]{Innovation and Technology in Computer Science Education}{July 2--4, 2018}{Larnaca, Cyprus}

\title{An Interactive Online Course to Raise Awareness about Diversity}

\author{Guido R\"o\ss{}ling}
\email{roessling.gu@kiva.tu-darmstadt.de}
\affiliation[obeypunctuation=true]{
  \institution{TU Darmstadt, CS Dept.}\\
  \streetaddress{Hochschulstr. 10}
  \postcode{64289}
  \city{Darmstadt},
  \country{Germany}
}
\author{Felix Broj}
\email{broj.fe@kiva.tu-darmstadt.de}
\affiliation[obeypunctuation=true]{
  \institution{TU Darmstadt, Pedagogy Dept.}\\
  \streetaddress{Alexanderstr. 23}
  \postcode{64289}
  \city{Darmstadt},
  \country{Germany}
}
\begin{CCSXML}
<ccs2012>
<concept>
<concept_id>10003456.10003457.10003527.10003531.10003533</concept_id>
<concept_desc>Social and professional topics~Computer science education</concept_desc>
<concept_significance>500</concept_significance>
</concept>
</ccs2012>
\end{CCSXML}

\ccsdesc[500]{Social and professional topics~Computer science education}
\keywords{online course, interactive, self-training, diversity, awareness}

\begin{abstract}
Slide-based lectures are not the optimal approach for learning new content or rehearsing materials for
all students. An alternative is using online courses, which allow students to proceed at their own pace.
We discuss the development  that allow interactivity and thus give students a chance to
actively participate and to easily measure their level of understanding. 
\end{abstract}

\maketitle

\section{An Interactive Course about Diversity}

Online materials of different kinds have become a common addition to current teaching. However, in
many cases, the online materials are non-interactive, e.g., slide copies, lecture recordings, YouTube
videos. This causes the students accessing the materials to be mostly passive, as they have no
proper way to adapt the materials to their level of understanding, apart from skipping ahead or pausing
the materials as needed.

Interactive course materials appear to be hard and costly, in terms of invested time and money, to
produce. In many cases, this impression is mostly correct, as software that can make content lively
and interesting, such as GoAnimate for creating short videos with animated characters \cite{GoAnimate},
often has a limited scope but already a moderately high price. Other approaches, for example the quiz options
integrated into current learning management systems such as Moodle \cite{Moodle}, are too limited in
scope to allow much flexibility. More advanced features, such as the Moodle ``lesson'' actiivity which
allows branching depending on the user's answer to questions, can easily lead to dead-ends or unexpected
results if multiple branches are used.

After some research, we found a tool that addressed these shortcoming: Articulate Storyline
360 \cite{Storyline}. With a GUI that is close to Microsoft PowerPoint in appearance, learning to use
Storyline to build a ``story'' out of slides is very easy. Beyond the functionality of PowerPoint, Storyline 360
offers the following (incomplete) set of interesting features for interactive courses:

\begin{itemize}
\item boolean, numeric, and String variables to remember past actions or input, such as the user's current 
name, current score, or last action(s) chosen,
\item arbitrary branching possible on each slide, based on different user actions (for example clicking on a button), or the current values of variables,
\item each slide can have an arbitrary number of layers that overlay the basic layer with (different) pieces of
information and reduce the number of slides necessary,
\item a decent ``scene view'' in which the structure of a story is visualized and the branching is clearly
shown,
\item photographic or illustrated characters in a large set of different poses and with different facial expressions and gestures,
\item the ability to modify a character's state based on user actions or variable values, for example by
turning a character from a smiling expression into a disapproving expression if a ``bad choice'' was made,
providing additional visual clues to the quality of the user's choice,
\item a large set of predefined automatically graded quizzes, including drag and drop as well as ``free form''
quizzes based on the correct arrangement of arbitrary screen elements,
\item the ability to record the screen and use this recoding as a direct playback, a segmented tutorial
video, or an interactive ``click after me'' tutorial (with automatically generated hints).
\end{itemize}

%\setcitestyle{acmnumeric}

\begin{acks}
This work was funded by the German Federal Ministry of Education and Research as part of the 
``Competence Development through Interdisciplinary and International Cooperation from the Start
(K$\mathrm{I^2}$VA)'' project.
\end{acks}

\bibliographystyle{ACM-Reference-Format}
\bibliography{iticse18}
\end{document}
