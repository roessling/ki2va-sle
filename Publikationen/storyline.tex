	%\documentclass[sigconf,anonymous]{acmart}
\documentclass[sigconf]{acmart}
\begin{document}

\acmConference[ITiCSE'18]{Innovation and Technology in Computer Science Education}{July 2--4, 2018}{Larnaca, Cyprus}

\title{Creating Interactive Online Courses With Articulate 360}
\setcopyright{rightsretained}

\author{Guido R\"o\ss{}ling}
\email{roessling@acm.org}
\affiliation[obeypunctuation=true]{
  \institution{TU Darmstadt, CS Dept.}\\
  \streetaddress{Hochschulstr. 10}
  \postcode{64289}
  \city{Darmstadt},
  \country{Germany}
}
\author{Felix Broj}
\email{broj.fe@kiva.tu-darmstadt.de}
\affiliation[obeypunctuation=true]{
  \institution{TU Darmstadt, Pedagogy Dept.}\\
  \streetaddress{Alexanderstr. 23}
  \postcode{64289}
  \city{Darmstadt},
  \country{Germany}
}
\begin{CCSXML}
<ccs2012>
<concept>
<concept_id>10003456.10003457.10003527.10003531.10003533</concept_id>
<concept_desc>Social and professional topics~Computer science education</concept_desc>
<concept_significance>500</concept_significance>
</concept>
</ccs2012>
\end{CCSXML}

\ccsdesc[500]{Social and professional topics~Computer science education}
\keywords{Articulate 360, online course, interactive, self-training}

\begin{abstract}
Slide-based lectures are not the optimal approach for learning new content or rehearsing materials for
all students. An alternative is using online courses, which allow students to proceed at their own pace.
We discuss a software package that allows flexible interactivity and thus gives students a chance to
actively participate and to easily measure their level of understanding. 
\end{abstract}

\maketitle

\section{Creating Interactive Online Courses Using Articulate Storyline}

Online materials of different kinds have become a common addition to current teaching. However, many
online materials are non-interactive, for example copies, lecture recordings or YouTube
videos. This may cause the students accessing the materials to be mostly passive, and thus less
engaged, as they have no proper way to adapt the materials to their level of understanding,
apart from skipping ahead or pausing the materials as needed.

Creating interactive course materials may appear to be a daunting challenge, in terms of invested
time and money and perceived required skills. Indeed, software that can
make content lively and interesting, such as GoAnimate \cite{GoAnimate} for creating short videos with animated
characters, is typically either expensive, hard to use, does not support meaningful decisions and interactivity, or
combines some of these drawbacks. Other approaches, for example the quiz options integrated into curren
learning management systems such as Moodle \cite{Moodle}, are too limited to allow much flexibility. More advanced
features, such as the Moodle ``lesson'' activity which allows branching depending on the user's answer
to questions, are time-intensive and complex to create and maintain. Preventing dead-ends in the navigation can also be
difficult: for example, the Moodle ``lesson'' provides no visualization of the ``branching navigation graph''.

Based on our experience, many of these issues can be overcome when using Articulate 360
\cite{Storyline}, a commercial tool with a yearly subscription price of currently \$499 per user
with an academic discount. With a GUI that is similar to Microsoft PowerPoint, learning to use
Storyline to build a ``story'' out of existing slides or from scratch is not hard. Beyond the functionality
of PowerPoint, Storyline 360 offers the following (incomplete) set of interesting features for
interactive courses that we have used to build several course units for self-study purposes:

\begin{itemize}
\item boolean, numeric, and String variables to remember past actions or input, such as the user's  
name, current score, or last action(s) chosen,
\item arbitrary branching is possible on each slide, based on different user actions (such as
clicking on a concrete button), or the current values of variables,
\item each slide can have an arbitrary number of layers that overlay the basic layer with (different) pieces of
information and reduce the number of required slides,
\item a decent ``scene view'' in which the structure of a story is visualized and the branching is clearly
shown,
\item photographic or illustrated characters in a large set of different poses and with different
facial expressions and gestures,
\item the ability to modify a character's state based on user actions or variable values. For example,
a ``bad choice'' can be used to give the character a disapproving expression,
providing additional visual clues to the quality of the user's choice,
\item a large set of predefined automatically graded quizzes, including drag and drop as well as ``free form''
quizzes based on the correct arrangement of arbitrary screen elements,
\item the ability to record the screen and use this recoding as a direct playback, a segmented tutorial
video, or an interactive ``click after me'' tutorial (with automatically generated hints).
\item the courses can be embedded on a web server, inside an LMS such as Moodle \cite{Moodle}
or published on a web-based reviewing platform where any user can comment on the contents if the
link was shared and they register with their mail address.
\end{itemize}

\begin{acks}
This work was funded by the German Federal Ministry of Education and Research as part of the 
``Competence Development through Interdisciplinary and International Cooperation from the Start
(K$\mathrm{I^2}$VA)'' project at the TU Darmstadt, Germany.
\end{acks}

\bibliographystyle{ACM-Reference-Format}
\bibliography{iticse18}
\end{document}
